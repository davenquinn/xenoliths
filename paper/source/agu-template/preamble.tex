%% 11/23/2015
%%%%%%%%%%%%%%%%%%%%%%%%%%%%%%%%%%%%%%%%%%%%%%%%%%%%%%%%%%%%%%%%%%%%%%%%%%%%
% AGUJournalTemplate.tex: this template file is for articles formatted with LaTeX
%
% This file includes commands and instructions
% given in the order necessary to produce a final output that will
% satisfy AGU requirements. 
%
% You may copy this file and give it your
% article name, and enter your text.
%
%%%%%%%%%%%%%%%%%%%%%%%%%%%%%%%%%%%%%%%%%%%%%%%%%%%%%%%%%%%%%%%%%%%%%%%%%%%%
% PLEASE DO NOT USE YOUR OWN MACROS
% DO NOT USE \newcommand, \renewcommand, or \def, etc.
%
% FOR FIGURES, DO NOT USE \psfrag or \subfigure.
% DO NOT USE \psfrag or \subfigure commands.
%%%%%%%%%%%%%%%%%%%%%%%%%%%%%%%%%%%%%%%%%%%%%%%%%%%%%%%%%%%%%%%%%%%%%%%%%%%%
%
% Step 1: Set the \documentclass
%
% There are two options for article format:
%
% 1) PLEASE USE THE DRAFT OPTION TO SUBMIT YOUR PAPERS.
% The draft option produces double spaced output.
% 
% 2) numberline will give you line numbers.

%% To submit your paper:
\documentclass[draft,numberline]{agujournal}

%% For final version.
% \documentclass{agujournal}

% Now, type in the journal name: \journalname{<Journal Name>}
\usepackage{siunitx}
\usepackage[version=3]{mhchem}
\usepackage{amsmath,xfrac,nicefrac}
\usepackage{paralist}
\usepackage{enumitem}
\usepackage{longtable}
\usepackage{multirow}
\usepackage{adjustbox}
\usepackage{array,booktabs,tabularx}
\usepackage{tablefootnote}
\usepackage[flushleft]{threeparttable}
%
% JGR-Atmospheres
% JGR-Biogeosciences
% JGR-Earth Surface
% JGR-Oceans
% JGR-Planets
% JGR-Solid Earth
% JGR-Space Physics
% Global Biochemical Cycles
% Geophysical Research Letters
% Paleoceanography
% Radio Science
% Reviews of Geophysics
% Tectonics
% Space Weather
% Water Resource Research
% Geochemistry, Geophysics, Geosystems
% Journal of Advances in Modeling Earth Systems (JAMES)
% Earth's Future
% Earth and Space Science
%
%

\journalname{Geochemistry, Geophysics, Geosystems}


\begin{document}

%% ------------------------------------------------------------------------ %%
%  Title
% 
% (A title should be specific, informative, and brief. Use
% abbreviations only if they are defined in the abstract. Titles that
% start with general keywords then specific terms are optimized in
% searches)
%
%% ------------------------------------------------------------------------ %%

\title{Late-Cretaceous construction of the mantle lithosphere beneath
the central California coast revealed by Crystal Knob xenoliths}

\authors{D. P. Quinn\affil{1}, J. Saleeby\affil{1},
         M. Ducea\affil{2,3}, P. Luffi\affil{4}, P. Asimow\affil{5}}

\affiliation{1}{Division of Geological and Planetary Sciences,
  California Institute of Technology, Pasadena, California, USA}
\affiliation{2}{Department of Geosciences, University of Arizona, Tuscon, Arizona, USA}
\affiliation{3}{Universitatea Bucuresti, Facultatea
          de Geologie Geofizica, Strada N. Balcescu Nr 1,
          Bucuresti, Romania}
\affiliation{4}{Institute of Geodynamics, Romanian Academy, 19-21
         Jean-Luis Calderon St., Bucharest, Romania}

\correspondingauthor{D. P. Quinn}{davenquinn@caltech.edu}



