title: Probing the origin of the sub-Salinia mantle lithosphere using
spinel lherzolite xenoliths from Crystal Knob, Santa Lucia Range,
California

\begin{enumerate}
\def\labelenumi{\arabic{enumi}.}
\item
  Mantle lithosphere is important for understanding processes,
  especially on western margin
\item
  Xenoliths help constrain mantle lithosphere makeup
\item
  We have them (first time for this region)
\item
  Regional makeup - Salinian block - used to be offshore of Mojave
\item
  Several hypotheses about how this mantle got here

  \begin{itemize}
  \itemsep1pt\parskip0pt\parsep0pt
  \item
    Farallon plate (seen further inland)
  \item
    Monterey plate
  \item
    upwelling asthenosphere from slab window
  \item
    Continental mantle wedge (unlikely)
  \end{itemize}

  These imply different thermal scenarios as well as rock compositions
  (mantle wedge) Xenoliths give us the best chance yet of answering this
  question
\end{enumerate}

\section{Methods}\label{methods}

\begin{itemize}
\itemsep1pt\parskip0pt\parsep0pt
\item
  Mineral modes
\item
  Major elements
\item
  Trace elements
\item
  Isotopes
\end{itemize}

\section{Results}\label{results}

\subsection{Petrography and mineral
modes}\label{petrography-and-mineral-modes}

\subsection{Thermometry}\label{thermometry}

\subsection{Trace elements}\label{trace-elements}

\subsection{Isotopes on clinopyroxene
separates}\label{isotopes-on-clinopyroxene-separates}

\section{Discussion}\label{discussion}

\subsection{Three emplacement options}\label{three-emplacement-options}

\begin{enumerate}
\def\labelenumi{\arabic{enumi}.}
\item
  Slab window \citet{Saleeby2003}

  \begin{itemize}
  \itemsep1pt\parskip0pt\parsep0pt
  \item
    Possible depending on depth of Monterey slab breakoff
  \end{itemize}
\item
  Monterey plate

  \begin{itemize}
  \itemsep1pt\parskip0pt\parsep0pt
  \item
    Possible (Monterey plate is definitely there)
  \item
    Far back from subduction interface
  \end{itemize}
\item
  Farallon plate

  \begin{itemize}
  \itemsep1pt\parskip0pt\parsep0pt
  \item
    This would conform with findings of Luffi et al
  \end{itemize}
\end{enumerate}

These options imply differences in thermal history

\begin{itemize}
\itemsep1pt\parskip0pt\parsep0pt
\item
  Thermal modeling
\end{itemize}
