\begin{tabularx}{\textwidth}{l l X}
\toprule
  Sample & Type & Notes \\
\midrule
  CK-1 & basalt & Host lava containing phenocrysts, xenocrysts, and < 1 cm xenolith fragments [Figure \ref{fig:microscope-images}a] \\
  CK-2 & lhz & Fertile lherzolite with pristine textural features \\
  CK-3 & hzb & Depleted harzburgite with the largest crystals in the sample set, small equant spinels \\
  CK-4 & hzb & Most affected by post-formation melting, with grain-boundary melt veins containing
               10-50 \si{um} aggregates of equant amphibole and other phases \\
  CK-5 & hzb & Relatively depleted sample, but with more clinopyroxene than CK-3 \\
  CK-6 & lhz & Fertile, with abundant spinel, large intergrown crystals, and recrystallized aggregates of orthopyroxene and clinopyroxene. Small-scale,
               vermicular intergrowth of clinopyroxene within orthopyroxene. \\
  CK-7 & lhz & Somewhat altered, with small-volume melt veins, but without growth of microcrystalline aggregates (as in CK-4). \\
  CK-D1 & dun & Host lava containing dunite and peridotite xenolith fragments. Peridotite orthopyroxenes contains graphic exsolution
               lamellae of  clinopyroxene within orthopyroxene [Figure \ref{fig:microscope-images}c] \\
  CK-D2 & dun & Host lava containing dunite and peridotite xenolith fragments \\
\bottomrule
\end{tabularx}
