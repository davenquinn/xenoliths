<# macro Ar(iso) -#>
  \ce{^{<<iso>>}Ar}
<#- endmacro #>

<# macro ratio(top,btm) -#>
  $\displaystyle\frac{\mbox{<<top>>}}{\mbox{<<btm>>}}$
<#- endmacro #>

<# macro multicell(main, after="") -#>
  \multirow{2}{*}{<<main>> \footnotesize <<after>>}
<#- endmacro #>

<# macro unitcell(main, type="c") -#>
  \multicolumn{1}{<<type>>}{\footnotesize <<main>>}
<#- endmacro #>

<# macro h(main) -#>
  \multicolumn{1}{c}{<<main>>}
<#- endmacro #>

{
\begin{tabular}{l l <<"r "*10 >>} %S[table-format=3.2]
\toprule
  Step &
  Power &
  << multicell(ratio(Ar(36),Ar(39)), "\si{\\times 10^{3}}") >> &
  << multicell(ratio(Ar(37),Ar(39))) >> &
  << multicell(ratio(Ar(38),Ar(39)), "\si{\\times 10^{2}}") >> &
  << multicell(ratio(Ar(40),Ar(39))) >> &
  << h(Ar(39)) >> &
  << h(Ar(39)) >> &
  << multicell(ratio("\ce{Ca}","\ce{K}")) >> &
  << multicell(ratio(Ar(40)+"*",Ar(39))) >> &
  << h(Ar(40)+"*") >> &
  << h("Age") >> \\
  &
  << unitcell("\si{W}") >> &
  &
  &
  &
  &
  << unitcell("\si{mol \\times 10^{-12}}","r") >> &
  << unitcell("\%") >> & & &
  << unitcell("\%") >> &
  << unitcell("\si{Ma}") >> \\
\midrule
<#- for ix, r in data  #>
  << loop.index >> &
  << ix|f('{:4}') >> &
  << r[0]|unp(2) >> &
  << r[1]|unp(2) >> &
  << r[2]|unp(2) >> &
  << r[3]|unp(2) >> &
  << r[4]|n(2) >> &
  << r[5]|n(3) >> &
  << r[6]|unp(2) >> &
  << r[7]|unp(2) >> &
  << r[8]|n(2) >> &
  << r[9]|n(2) >> \\
<#- endfor  #>
\bottomrule
\end{tabular}
}
